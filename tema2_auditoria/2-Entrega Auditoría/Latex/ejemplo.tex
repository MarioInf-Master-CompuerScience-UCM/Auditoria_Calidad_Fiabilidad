\input{preambuloProblemas}

\usepackage{pgf}
\usepackage{tikz}
%\usepackage{pstricks,pst-node,pst-tree,pstricks-add}
\usepackage{url}
\usepackage{hyperref}
\usepackage{xcolor,colortbl}

\definecolor{Gray}{gray}{0.85}

\newcommand{\ojos}{%
  \unitlength=0.7mm
  \special{em:linewidth 0.4pt}
  \linethickness{0.4pt}
  \begin{picture}(6.00,4.00)
    \put(2.00,2.00){\circle{4.00}}
    \put(6.00,2.00){\circle{4.00}}
    \put(3.00,2.00){\circle*{2.00}}
    \put(7.00,2.00){\circle*{2.00}}
  \end{picture}
}%ojos

\newcommand{\OJO}[1]{%
   \ojos\ {\bf [}{#1}{\bf ]}
}%OJO

% Para el checklist
\usepackage{wasysym}     

\setlength{\marginparwidth}{1.2in}
\let\oldmarginpar\marginpar
\renewcommand\marginpar[1]{\-\oldmarginpar[\raggedleft #1]%
{\raggedright #1}}    

\newenvironment{checklist}{%
  \begin{list}{}{}% whatever you want the list to be
  \let\olditem\item
  \renewcommand\item{\olditem -- \marginpar{$\Box$} }
  \newcommand\checkeditem{\olditem -- \marginpar{$\CheckedBox$} }
}{%
  \end{list}
}   

\begin{document}

\begin{center}
{\Large\textbf{Auditor\'ia, Calidad y Fiabilidad Inform\'aticas}}
\smallskip

\large
Universidad Complutense de Madrid \qquad Curso 2022/2023
\medskip

\end{center}

\title{Informe de auditor\'ia de la aplicaci\'on XXX}
\author{Adri\'an Riesco Rodr\'iguez}
\date{}
\maketitle

%\section*{Objetivos del informe de auditor\'ia para\ldots}

\begin{center}

%\section*{Datos del auditor}
%
%\begin{itemize}
%\item
%\texttt{Nombre y apellidos:}
%
%\item
%\texttt{Proyecto a auditar:}
%\end{itemize}

%\section*{Documentaci\'on a entregar}
%
%\begin{checklist}
%\item
%Datos del auditor
%
%\item
%Documentaci\'on analizada
%
%\item
%Entrevistas realizadas
%
%\item
%Alcance
%
%\item
%Informe
%\end{checklist}

%\newpage

\section*{Destinatario}

El destinatario de este informe es XXX.

\OJO{El destinatario no tiene por qu\'e ser la entidad auditada; puede
haber sido un tercero.}

\section*{Entidad auditada}

La entidad auditada es XXX. Esta empresa de desarrollo de software est\'a
especializada en aplicaciones web para YYY. En particular, se audita el
proyecto ZZZ desarrollado para el cliente CCC.

\section*{Alcance}

Como veremos en la secci\'on de objetivos, en esta auditor\'ia vamos a
analizar los siguientes puntos:

\begin{itemize}
\item
Contrato, estudiando c\'omo se gestionaron los requisitos.

\item
Plan de desarrollo, fij\'andonos especialmente en el plan de desarrollo,
las tareas cr\'iticas y los hitos.

\item
\OJO{Continuar hasta describir todos los puntos generales.}
\end{itemize}

\section*{Comparabilidad}

No se dispone de informaci\'on sobre informes de auditor\'ia anteriores.

\OJO{Por supuesto, si hay informes previos explicarlos, dando fechas,
indicando qui\'en hizo la auditor\'ia, los resultados y la posible relaci\'on
con la auditor\'ia actual}

\section*{Salvedades}

Se han detectado las siguientes salvedades:
\begin{itemize}
\item
No se pudo contactar con los desarrolladores, por lo que no se pudieron
confirmar ciertos aspectos poco desarrollados en la memoria.

\item
\OJO{Continuar con todos los problemas surgidos durante la auditor\'ia.}
\end{itemize}

\section*{Incumplimientos}

\begin{tabular}{|l|c|c|c|c|c|}
\hline
\texttt{Item} & \texttt{Objetivo} & \texttt{S\'i} & \texttt{No} & \texttt{Parcial} &
\texttt{N.A.}\\
\hline
\rowcolor{Gray}
1. & Contrato &&&&\\
\hline
1.1. & Se reflejan los requisitos & X &&&\\
\hline
1.2. & Se justifican los requisitos rechazados && X &&\\
\hline
1.3. & Se tienen en cuenta los riesgos &&& X &\\
\hline
\rowcolor{Gray}
2. & Plan de desarrollo &&&&\\
\hline
2.1. & Plan de desarrollo definido & X &&&\\
\hline
2.2. & Plan de desarrollo documentado & X &&&\\
\hline
2.3. & Tareas cr\'iticas establecidas & X &&&\\
\hline
2.4. & Hitos establecidos & X &&&\\
\hline
\rowcolor{Gray}
3. & Requisitos &&&&\\
\hline
3.1. & Requisitos claramente definidos & X &&&\\
\hline
3.2. & Requisitos claramente documentados & X &&&\\
\hline
3.3. & Existen requisitos de mantenibilidad &&&&\\
\hline
\rowcolor{Gray}
4. & Dise\~no &&&&\\
\hline
4.1. & Dise\~no claramente documentado & X &&&\\
\hline
4.2. & Revisiones formales de dise\~no & X &&&\\
\hline
4.3. & Revisiones por pares & X &&&\\
\hline
4.4. & Revisiones por expertos &&&& X\\ 
\hline
\rowcolor{Gray}
5. & Implementaci\'on &&&&\\
\hline
5.1. & Pruebas de unidad & X &&&\\
\hline
5.2. & Testing documentado & X &&&\\
\hline
5.3. & Revisiones por pares & X &&&\\
\hline
\rowcolor{Gray}
6. & Seguimiento &&&&\\
\hline
6.1. & Hitos seguidos & X &&&\\
\hline
6.2. & Riesgos seguidos & X &&&\\
\hline
6.3. & Riesgos solucionados && X &&\\
\hline
\rowcolor{Gray}
7. & Documentaci\'on &&&&\\
\hline
7.1. & Documentos controlados && X &&\\
\hline
7.2. & Responsable de documentos controlados && X &&\\
\hline
7.3. & Documentaci\'on para testing & X &&&\\
\hline
7.4. & Documentaci\'on de instalaci\'on y uso &&& X &\\
\hline
\end{tabular}
\end{center}

Evaluaci\'on detallada:
\begin{itemize}
\item[1.1] Se reflejan los requisitos, como se puede ver
en las p\'aginas PPP del documento DDD y en las p\'aginas XXX
del documento YYY.

\item[1.2] No se tiene constancia alguna de justificaci\'on
de los requisitos rechazados.

\textbf{Recomendaci\'on:} El uso de documentos controlados y
el nombramiento de un responsable facilitar\'ia el seguimiento
de los cambios, por lo que se recomienda seguir estos pasos e
incluir el contrato como parte de los documentos controlados.

\OJO{Dar la recomendaci\'on ``justificar los requisitos
rechazados''  es como no decir nada.}

\item[1.3] Aunque los riesgos se mencionan el la p\'agina
XXX del documento YYY, no hay ninguna otra menci\'on en el resto
de la documentaci\'on, por lo que se considera que este punto
no es completamente v\'alido.

\textbf{Recomendaci\'on:} Como se indic\'o en el anterior punto\ldots
\OJO{completar}

\item[4.4.] No se da ninguna informaci\'on en la documentaci\'on sobre
revisiones por expertos. Sin embargo, dado que la aplicaci\'on XXX es
suficientemente sencilla y la empresa YYY cuenta con sobrada experiencia
en el desarrollo de este tipo de aplicaciones, \textbf{como se indic\'o en la
explicaci\'on de la entidad auditada}, consideramos que no era relevante
en este caso.

\item[X.Y] \OJO{Explicar todos los puntos. Todos los puntos necesitan explicaciones y aquellos marcados como \textbf{No} o \textbf{Parcial} necesitan recomendaciones.
Es necesario siempre hacer referencia a p\'aginas y documentos concretos para
justificar los puntos.}
\end{itemize}

\section*{Enf\'asis}

El problemas m\'as grave es la mala gesti\'on del contrato, que puede suponer
graves problemas a lo largo del todo el proyecto. El contrato es un documento
clave y de su buena gesti\'on depende en buena parte el \'exito del proyecto,
por lo que debe prestarse especial atenci\'on a la fase de creaci\'on y revisi\'on.
Asimismo, es recomendable que se reflejen adecuadamente los riesgos para justificar
costes y fechas.

\OJO{De todos los incumplimientos elegimos y comentamos los m\'as graves.}

\section*{Informe de gesti\'on}

En esta auditor\'ia no hemos tenido acceso a documentaci\'on sobre la gesti\'on.

\OJO{Aunque esta parte es importante en general, en nuestro caso no tenemos acceso
a la documentaci\'on. En general, veremos c\'omo gestionan los jefes los recursos,
tanto humanos como hardware/software.}

\section*{Resumen}

Hemos encontrado incumplimientos en el contrato, el seguimiento y la
documentaci\'on. Estos incumplimientos parecen no haber tenido impacto durante el
desarrollo, pues el producto se entreg\'o a tiempo y sin defectos. Sin embargo,
puede tener efectos durante el mantenimiento, ya que\ldots \OJO{Continuar}

Por ello, recomendamos\ldots \OJO{Continuar}

\section*{Resultado}

{\huge Favorable CON SALVEDADES}

\vspace{3ex}

Se recomienda revisar los procedimientos para el contrato y la documentaci\'on,
tal y como se ha indicado en los puntos anteriores.

\section*{Fecha y firma}

Firmado a \today\ en Madrid

\vspace{10ex}

Adri\'an Riesco Rodr\'iguez

\end{document}
